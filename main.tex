% --------------------------------------------------------------------
% --------------------------------------------------------------------

% Modelo para dissertações e teses do IESP-UERJ, adaptado da classe repUERJ
% do Prof. Luís Fernando de Oliveira (IF-UERJ). 
%
% O modelo se apresenta como uma solução personalizada para o Instituto
% de Estudos Sociais e Políticos da UERJ, a pedidos da biblioteca, sob 
% orientação de Rosalina Maria de Sousa Barros, Chefe da Biblioteca CCS/D - IESP. 
%
% O modelo está mais enxuto que o original, e com comentários sobre
% a formatação.

% O código original está disponível no site do Prof. Luís Fernando:
%      http://sites.google.com/site/deoliveiralf
%
% As normas da UERJ para elaboração de teses e dissertações pode ser
% obtidas no documento disponível no site
%      http://www.bdtd.uerj.br/roteiro_uerj_web.pdf
%
%
%                                             Matheus Cavalcanti Pestana

% --------------------------------------------------------------------
% --------------------------------------------------------------------
%
\documentclass[a4paper,12pt,oneside,onecolumn,final,fleqn]{repUERJ}
% ---
% Pacotes fundamentais 
% ---
\usepackage[brazil]{babel}  % adequação para o português Brasil
\usepackage[utf8]{inputenc} % Determina a codificação utilizada
                            % (conversão automática dos acentos)
\usepackage{makeidx}        % Cria o índice
\usepackage{hyperref}       % Controla a formação do índice
\usepackage{indentfirst}    % Endenta o primeiro paragrafo de
                            % cada seção.
\usepackage{graphicx}       % Inclusão de gráficos
\usepackage{subfig}
\usepackage{amsmath}        % pacote matemático
\usepackage{subfiles}       % pacote para inserir sub-arquivos no LaTex
\usepackage{booktabs}       % pacote para tabelas melhores
\usepackage{caption}        % pacote para captions em objetos
\usepackage{longtable}      % pacote para tabelas de mais de uma página
\usepackage{pdflscape}      % pacote para rotacionar o pdf (ambiente pdflscape)
\usepackage{gensymb}        % pacote para alguns símbolos, como º,ª, etc
\usepackage{makecell}       % pacote para melhor personalização de tabelas
\usepackage{blindtext}      % para criar texto aleatório
% ---
% Pacote auxiliar para as normas da UERJ
% ---
\usepackage[frame=no,algline=yes,font=times]{repUERJformat} % Adaptei para Times
\usepackage{repUERJpseudocode}
% ---
% Pacotes de citacoes
% ---
\usepackage[alf]{abntex2cite}

% ********************************************************************
% ********************************************************************
% Informações de autoria e institucionais
% ********************************************************************
% ********************************************************************

%---------------------------------------------------------------------
% Imagens pretextuais (precisam estar no mesmo diretório deste arquivo .tex)
%---------------------------------------------------------------------

\logo{logo_uerj_cinza.png}
\marcadagua{marcadagua_uerj_cinza.png}{1}{160}{255}

%---------------------------------------------------------------------
% Informações da instituição
%---------------------------------------------------------------------

\instituicao{Universidade do Estado do Rio de Janeiro}
            {Centro de Ciências Sociais} 
            {Instituto de Estudos Sociais e Políticos} 
            %{patrono} 

%---------------------------------------------------------------------
% Informações da autoria do documento
%---------------------------------------------------------------------

\autor{Matheus}
      {Cavalcanti Pestana}
      {M.}

\titulo{Título do trabalho acadêmico}
% se não for usar a quarta palavra chave, deixar o campo vazio: {}
% O modelo do Prof. Luís Fernando só deixa espaço para quatro palavras chaves. 
% Todavia, pelas normas da UERJ, não há limite. 

\palavraschaves{Primeira palavra chave}
               {Segunda palavra chave}
               {Terceira palavra chave}
               {}

\title{Title of dissertation}
\keywords{First keyword}
         {Second keyword}
         {T hird keyword}
         {}

\orientador{Cargo Titulação}  % Prof. Dr. 
           {Nome}{Sobrenome} 
           {Instituto de Estudos Sociais e Políticos -- UERJ}  % Sempre utilizar -- para o "meia-risca"

%coorientador é opcional
%\coorientador{Cargo Titulação} 
%             {Nome}{Sobrenome} 
%             {IESP -- UERJ} % Sempre utilizar -- para o "meia-risca"

%---------------------------------------------------------------------
% Grau pretendido (Doutor, Mestre, Bacharel, Licenciado) e Curso
%---------------------------------------------------------------------

\grau{Doutor}  % Doutor ou Mestre (ou flexionar no gênero feminino)
\curso{Ciência Política} % Ciência Política ou Sociologia

% área de concentração é opcional
%\areadeconcentracao{área}

%---------------------------------------------------------------------
% Informações adicionais (local, data e paginas)
%---------------------------------------------------------------------

\local{Rio de Janeiro} 
\data{dd}{mês}{aaaa} 

% ********************************************************************
% ********************************************************************
% Configurações de aparência do PDF final
% ********************************************************************
% ********************************************************************

% alterando o aspecto da cor azul
\definecolor{blue}{RGB}{41,5,195}
%\definecolor{apricot}{RGB}{251,206,177}

% informações do PDF
\hypersetup{
  %backref=true,
  %pagebackref=true,
  %bookmarks=true,
  unicode=false,
  pdftitle={\UERJtitulo},
  pdfauthor={\UERJautor},
  pdfsubject={\UERJpreambulo},
  pdfkeywords={PALAVRAS}{CHAVES}{\chaveA}{\chaveB}{\chaveC}{\chaveD},
  pdfproducer={\packagename},       % producer of the document
  pdfcreator={\UERJautor},
  colorlinks=true,       % false: boxed links; true: colored links
  linkcolor=black,       % color of internal links blue
  citecolor=black,       % color of links to bibliography blue
  filecolor=black,       % color of file links magenta
  urlcolor=black,
  bookmarksdepth=4
}

% ********************************************************************
% ********************************************************************
% Início do documento
% ********************************************************************
% ********************************************************************
% ---
% compila o índice; se não for usar, comentar
% ---
\makeindex
% ---
% ********************************************************************
% ********************************************************************
\begin{document}
% ----------------------------------------------------------
% ELEMENTOS PRE-TEXTUAIS
% ----------------------------------------------------------
\frontmatter
% ----------------------------------------------------------
% Capa e a folha de rosto
% ----------------------------------------------------------
\capa
\folhaderosto
% ----------------------------------------------------------
% Inserir a ficha catalográfica
% ----------------------------------------------------------

% A biblioteca deverá providenciar a ficha catalográfica. Salve a ficha no formato PDF.
% Use o nome do arquivo PDF como argumento do comando. Exemplo: ficha catalográfica
% no arquivo 'ficha.pdf'

%\fichacatalografica{ficha.pdf}

% Enquanto não possuir a ficha catalográfica, use o comando sem argumentos...

\fichacatalografica{}

% ----------------------------------------------------------
% Folha de aprovação
% ----------------------------------------------------------
% membros da banca: máximo 6

% Comentário sobre a folha de aprovação: 
% Caso o orientador não seja membro da banca, acima de "Banca Examinadora"
% deve constar "Orientador:", seguido do nome do mesmo, sem espaço para assinatura.
% Abaixo, "Banca Examinadora:", como presente, e com espaços de assinatura
% para cada membro, sem contar o orientador. 

\begin{folhadeaprovacao}
  \assinatura{primeiro membro titular da banca}{instituição}
  \assinatura{segundo membro titular da banca}{instituição}
  \assinatura{terceiro membro titular da banca}{instituição}
% suplente só é incluído se efetivamente substitui um titular
\end{folhadeaprovacao}
% ----------------------------------------------------------
% Dedicatória
% ----------------------------------------------------------
\pretextualchapter{Dedicatória}

\vfill
Texto da dedicatória
% ----------------------------------------------------------
% Agradecimentos
% ----------------------------------------------------------
\pretextualchapter{Agradecimentos}

Texto de agradecimento.

É OBRIGATÓRIO, para alunos que receberam bolsa da CAPES, agradecimentos à agência. Se tiver recebido bolsa por outra agência de fomento, é preciso pesquisar as regras para ver se é preciso a citação. 

O presente trabalho foi realizado com apoio da Coordenação de Aperfeiçoamento de Pessoal de Nível Superior Brasil (CAPES) - Código de Financiamento 001.

% ----------------------------------------------------------
% Epigrafe (opcional)
% ----------------------------------------------------------
\pretextualchapter{}

  \vfill\
  \begin{flushright}
    Texto da epígrafe
  \end{flushright}
% ----------------------------------------------------------
% RESUMO
% ----------------------------------------------------------
\pretextualchapter{Resumo}

\referencia

Texto do resumo em português.\\

\imprimirchaves
% ----------------------------------------------------------
% Abstract
% ----------------------------------------------------------
\pretextualchapter{Abstract}

\reference

Abstract in English.\\

\printkeys
% ----------------------------------------------------------
% Listas de ilustrações e tabelas
% ----------------------------------------------------------
\listadefiguras
\listadetabelas

\texttt{Segue o mesmo exemplo da lista de ilustrações} % Remover esta linha 
% ----------------------------------------------------------
% Outras listas
% ----------------------------------------------------------
% \listadealgoritmos
% ----------------------------------------------------------
% Lista de abreviaturas e siglas
% ----------------------------------------------------------
\pretextualchapter{Lista de abreviaturas e siglas}

\abreviatura{sigla1}{por extenso}
\abreviatura{sigla2}{por extenso}
\abreviatura{sigla3}{por extenso}
% ----------------------------------------------------------
% Lista de simbolos
% ----------------------------------------------------------
\pretextualchapter{Lista de símbolos}

\simbolo{simbolo1}{significado e/ou valor}
\simbolo{simbolo2}{significado e/ou valor}
\simbolo{simbolo3}{significado e/ou valor}
% ----------------------------------------------------------
% Sumario
% ----------------------------------------------------------
\sumario
% ----------------------------------------------------------
% ELEMENTOS TEXTUAIS
% ----------------------------------------------------------
\mainmatter
%=====================================================================
\chapter*{Introdução}
%=====================================================================

Texto da introdução. Texto, texto texto \cite{bib:Amado1991}, texto \citeonline{bib:Amado1991}. Texto \citeauthoronline{bib:Andrade1997} em \citeyear{bib:Andrade1997}, texto \citeauthor{bib:Andrade1997},  texto.

\Blindtext % Deletar essa linha para remover o texto em latim

%=====================================================================
\chapter{T\'itulo da seção 1}
%=====================================================================

Texto do capítulo. Texto, texto, Figura \ref{rotulo}. Texto Figura \ref{outro.rotulo}\subref{subrotulo1}.

Algumas orientações sobre figuras, tabelas, etc:
\begin{itemize}
    \item Se de elaboração própria, a referência logo abaixo do objeto deve se dar da seguinte forma: ``O autor, ano''. Exemplo: O autor, 2019.
    \item Se de elaboração de outro, a fonte deve vir ``SOBRENOME, ANO''. Exemplo: FIGUEIREDO, 1999. No \LaTeX, utilizar no campo \texttt{caption} a seguinte forma: 
    \begin{verbatim}
        \citeauthor{palavrachave}, \citeyear{palavrachave}.
    \end{verbatim}
    \item Tanto o título quanto a legenda devem estar alinhados com a figura, iniciando no mesmo ponto. 
    \item Se houver adatapção ou tradução de trechos citados, a referência a isso se dá da seguinte maneira:
    \begin{verbatim}
        \cite[p. NN, tradução do autor, ANO]{palavrachave}
    \end{verbatim}
    Exemplo: 
    \cite[p. 22, tradução do autor, 2019]{bib:Bernstein1997}\footnote{De acordo com a ABNT, é necessário sempre citar o ano da tradução, adaptação ou elaboração de fotografias, tabelas, ou seja, ilustrações em geral.}
    \item Equações devem ser alinhadas à esquerda, com numeração à direita. Exemplo:
    \begin{align}
    Gh & = \sqrt{[c \sum{(s_i - v_i)^2]}}
    \end{align}
    Tal formato só aparece com a utilização do ambiente \texttt{align}. Se houver apenas uma incidência de equação, a mesma não precisa ser numerada, e isso é obtido com o ambiente \texttt{align*}, com o asterisco. 
    \begin{verbatim}
        \begin{align}
             Gh & = \sqrt{[c \sum{(s_i - v_i)^2]}}
        \end{align}
        ou
        \begin{align*} % Para não numerar (incidência única)
             Gh & = \sqrt{[c \sum{(s_i - v_i)^2]}}
        \end{align*}
    \end{verbatim}
\end{itemize}


\section{Secundário}

% \begin{figure ou table}[posição]{largura da figura}

\begin{figure}[!hbpt]{6cm}
  \caption{Título da figura.} \label{rotulo}
  \includegraphics[width=\hsize]{logo_uerj_cor.jpg}
  \legend{Texto da legenda.}
  \source{Citação da fonte ou `O autor, 2019'.}
\end{figure}


\begin{figure}[!hbpt]{11cm}
  \caption{Título da figura.} \label{outro.rotulo}
  \subfloat[][]{\label{subrotulo1}
    \fbox{\includegraphics[width=0.45\hsize]{logo_uerj_cinza.png}}}\hfill
  \subfloat[][]{\label{subrotulo2}
    \fbox{\includegraphics[width=0.45\hsize]{marcadagua_uerj_cinza.png}}}\\
  \subfloat[][]{\label{subrotulo3}
    \fbox{\includegraphics[width=0.45\hsize]{logo_uerj_cor.jpg}}}\hfill
  \legend{Texto da legenda. \subref{subrotulo1} Texto da imagem;
          \subref{subrotulo2} Texto da imagem;
          \subref{subrotulo3} Texto da imagem.}
  \source{Citação da fonte ou `O autor, 2019'.}
\end{figure}

\subsection{Terciário}

\Blindtext % Deletar essa linha para remover o texto em latim

\subsubsection{Quaternário}

\Blindtext % Deletar essa linha para remover o texto em latim
%=====================================================================
\chapter{T\'itulo da seção 2}
%=====================================================================

\Blindtext % Deletar essa linha para remover o texto em latim

%=====================================================================
\chapter*{Conclusão}
%=====================================================================

Texto da conclusão.

% ----------------------------------------------------------
% ELEMENTOS POS-TEXTUAIS
% ----------------------------------------------------------
\backmatter
%=====================================================================
% Referencias via BibTeX
%=====================================================================
\citeoption{abnt-options4}
\bibliography{bibliografia}

    \textbf{Sobre as referências bibliográficas}: segundo a norma atual, o nome dos autores deve ser repetido, tantas as vezes quanto forem as obras citadas. Logo, o formato antigo, onde se lia:

    SANTOS, Wanderley Guilherme dos. Lorem ipsum...
    
    \_\_\_\_\_\_\_\_. Wanderley Guilherme dos. Dolor sit amet...
    
    Se torna:
    
    SANTOS, Wanderley Guilherme dos. Lorem ipsum...
    
    SANTOS, Wanderley Guilherme dos. Dolor sit amet...

    Atenção: em títulos de obras, somente a primeira palavra inicia com letra maiúscula. Todo o resto, exceto nomes próprios e nomes de países, iniciam com letra minúscula. Logo, o correto é: 
    
        Coronelismo, enxada e voto — o município e o regime representativo no Brasil
        
        \texttt{e não}
        
        Coronelismo, Enxada e Voto — O Município e o Regime Representativo no Brasil

%=====================================================================

%=====================================================================
\postextualchapter*{Glossário}
%=====================================================================
\definicao{termo}{significado}
\definicao{termo}{significado}
\definicao{termo}{significado}
% ----------------------------------------------------------
% Apêndices (opcionais)
% ----------------------------------------------------------
% ---
% Inicia os apêndices
% ---
\appendix
%=====================================================================
\postextualchapter{Primeiro apêndice}
%=====================================================================

%=====================================================================
\postextualchapter{Segundo apêndice}
%=====================================================================

% ----------------------------------------------------------
% Anexos (opcionais)
% ----------------------------------------------------------
% ---
% Inicia os anexos
% ---
\annex
%=====================================================================
\postextualchapter{Primeiro anexo}
%=====================================================================

%=====================================================================
\postextualchapter{Segundo anexo}
%=====================================================================

%---------------------------------------------------------------------
% INDICE REMISSIVO (relativo ao makeindex)
%---------------------------------------------------------------------
\printindex
%=====================================================================
\end{document}
